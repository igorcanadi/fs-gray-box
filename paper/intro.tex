For our project we chose to measure and compare several ext4 filesystem parameters
in an operating system (OS) running directly on a machine (host) and in the same 
OS running on a virtual machine (VM) on top of the host OS. In addition to being an 
interesting study in the inner workings of the ext4 filesystem, this paper seeks to
elucidate the similarities and differences between host and virtual machine I/O in
terms of latency, and also reveal the way I/O is virtualized in a VM.

In particular, we determined the ideal buffer size for random I/O by preforming random reads
of varying sizes. Next, we reverse engineered the prefetch size that the Linux kernel uses to speed up
sequential reads by measuring sequential read latency. We were also interested in the file cache size.
We measured the throughput of repeated reads of a big file, which decreased substantially as soon
as the file was bigger than the cache. Finally, we tried to determine the file size that triggers the
file system to add a layer of indirection between the inode and data blocks. By measuring the one-byte
append latency to a file, we expected to see higher latency when an append causes file system
to write additional metadata for an indirect pointer.

We found that the host and VM filesystem had similar characteristics across all measured parameters. For instance, the 
optimal size to randomly read a file was N*4KB, a multiple of the page size of the 
system, and that for large reads the best through put was achieved at approximately
7MB. The maximum prefetch size was 128KB for both the host and the VM. We also 
determined that the file cache size was related to the amount of physical memory 
available while our process was running. Finally, we proved that there is no single file
size for which ext4 adds another level of indirection in host environment. We could not prove a
similar hypothesis on the virtual machine.

The rest of the paper is organized as follows. In Section~\ref{sec:method} we 
discuss our experimental setup and methodology for our experiments. In 
Section~\ref{sec:results} we discuss our experiments and results in detail. 
Finally, in Section~\ref{sec:conc} we summarize our findings and conclusions.

