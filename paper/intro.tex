For our project we chose to measure and compare several ext4 filesystem parameters
in an operating system (OS) running directly on a machine (host) and in the same 
OS running on a virtual machine (VM) on top of the host OS. In addition to being an 
interesting study in the inner workings of the ext4 filesystem, this paper seeks to
elucidate the similarities and differences between host and virtual machine I/O in
terms of latency, and also reveal the way I/O is virtualized in a VM.

In particular, we examined the ideal buffer size for random I/O, sequential 
prefetch size, file cache size, and the file size which causes the file system 
to add an additional layer of indirection. We found that the host and VM filesystem
had similar characteristics across all measured parameters. For instance, the 
optimal size to randomly read a file was 4KB, the page size of the system, and 
the maximum prefetch size was 128KB for both the host and the VM. We also 
determined that the file cache was related to the amount of physical memory 
available while our process was running. Finally, we found that for both the host
and VM that the file size where another level of inodes is added was \#MB.

The rest of the paper is organized as follows. In Section~\ref{sec:method} we 
discuss our experimental setup and methodology for our experiments. In 
Section~\ref{sec:results} we discuss our experiments and results in detail. 
Finally, in Section~\ref{sec:conc} we summarize our findings and conclusions.

