We were able to measure the prefetch size read by the system during a sequential 
read. In Linux, there is a readahead algorithm that fetches a certain number of 
pages from disk given the read behavior. The set of pages in the cache fetched by 
the algorithm is called the readahead window. In a normal sequential read, the 
readahead algorithm starts with an initial read size based on the first read and in
subsequent requests grows the readahead window until it reaches the maximum size. 
The readahead algorithm marks a page at the lookahead index. When the application 
references this page, the algorithm grows the readahead window and asynchronously 
fetches the next set of pages from disk. The maximum window is determined by 
parameters in the kernel code, but it is typically 32 pages. In our system, we have
4KB pages so we expect the readahead size to grow exponentially and then plateau at
the maximum window size of 128KB. We expect the same for our virtual machine, which has the same readahead algorithm and page size.

Since the read latency is much greater when the requested page is not in the page 
cache, we can find the prefetch size by performing sequential reads and determining
the sizes in which spikes occur. We assume that the lookahead index is the last 
page of the readahead window. With this in mind, we wrote the code of 
Algorithm~\ref{alg:p2_code} to measure the prefetch size. In our code, we flush 
the cache before the initial read and measure the time to read READ\_SIZE number of
bytes at a time. We then wait 100ms to allow time for pages to be prefetched. This
eliminates false positives in the case where the read requests occur faster than
the data can be fetched from disk, resulting in a page fault.

%\begin{algorithmic}[h]
%	open file
%	flush cache
%	\for{i = 0 \to MAX_READS}
%	\state{
%	t1 = rdtsc_start()
%	read(READ_SIZE)	\comment{Read READ_SIZE bytes}
%	t2 = rdtsc_end()
%	}
%	wait 100ms
%	\endfor
%	close file
%	\caption{Pseudocode for measuring prefetch size}
%	\label{alg:p2_code}
%\end{algorithmic}

Figure~\ref{fig:p2_graph_big} shows the read latency for a READ\_SIZE of 512B with
respect to the total number of bytes read. We see spikes for the initial read, and
then at increasing intervals of 4KB, 12KB, 32KB, and 64KB, and then plateaus at 
128KB, following an exponential growth until it hits the maximum prefetch size as 
we expect.  

\begin{figure}[h]
	\includegraphics[width=0.5\textwidth]{./figures/p2_big.pdf}
	\caption{Time to read 512KB blocks with respect to total number of KB read sequentially from a file}
	\label{fig:p2_graph_big}
\end{figure}

\begin{figure}[h]
	\includegraphics[width=0.5\textwidth]{./figures/p2_small.pdf}
	\caption{Time to read 512KB blocks with respect to total number of KB read sequentially from a file}
	\label{fig:p2_graph_small}
\end{figure}




